%-------------------------
% Resume in Latex
% Author : Song Liu
% Adapted from: Indu dwivedi, Sourabh Bajaj
% License : MIT
%------------------------

\documentclass[letterpaper,10pt]{article}
\usepackage{latexsym}
\usepackage[empty]{fullpage}
\usepackage{titlesec}
\usepackage{marvosym}
\usepackage[usenames,dvipsnames]{color}
\usepackage{verbatim}
\usepackage{enumitem}
\usepackage[pdftex, hidelinks]{hyperref}
\usepackage{fancyhdr}
\usepackage[charter]{mathdesign} % Bitstream Charter
% \usepackage{newpxtext,newpxmath} % Palatino
\usepackage{longtable}
\usepackage{graphicx}
\usepackage{array}
\usepackage{multirow}
\usepackage{xcolor}
\usepackage{bibentry}
\pagestyle{fancy}
\fancyhf{} % clear all header and footer fields
\fancyfoot{}
\renewcommand{\headrulewidth}{0pt}
\renewcommand{\footrulewidth}{0pt}

% bibtex for publication
\bibliographystyle{plain}
\nobibliography{resume.bib}

% Adjust margins
\addtolength{\oddsidemargin}{-0.50in}
\addtolength{\evensidemargin}{-0.50in}
\addtolength{\textwidth}{1in}
\addtolength{\topmargin}{-.5in}
\addtolength{\textheight}{1.0in}

% Define colors
\definecolor{linkblue}{RGB}{111, 153, 222}
\definecolor{titleblue}{RGB}{46, 116, 181}
\urlstyle{same}

\raggedbottom
\raggedright
\setlength{\tabcolsep}{0in}

% Sections formatting
\titleformat{\section}{
  \vspace{-6pt}\scshape\raggedright\large
}{}{0em}{}[\color{black}\titlerule \vspace{-5pt}]

%-------------------------
% Custom commands
\newcommand{\resumeItem}[2]{
  \item\small{
    \textbf{#1}{: #2 \vspace{-2pt}}
  }
}

\newcommand{\resumeItemNoBullet}[2]{
  \item[]\small{
    \hspace{-9pt}\textbf{#1}{: #2 \vspace{-6pt}}
  }
}

\newcommand{\resumeSubheading}[4]{
  \vspace{-1pt}\item[]
  \begin{tabular*}{0.98\textwidth}{l@{\extracolsep{\fill}}r}
      \hspace{-10pt}\textbf{#1} & #2 \\
      \hspace{-10pt}\textit{\small#3} & \textit{\small #4} \\
    \end{tabular*}\vspace{-5pt}
}

\newcommand{\resumeSubItem}[2]{\resumeItem{#1}{#2}\vspace{-4pt}}

\renewcommand{\labelitemii}{$\circ$}

\newcommand{\resumeSubHeadingListStart}{\begin{itemize}[leftmargin=*]}
\newcommand{\resumeSubHeadingListEnd}{\end{itemize}}
\newcommand{\resumeItemListStart}{\begin{itemize}}
\newcommand{\resumeItemListEnd}{\end{itemize}\vspace{-5pt}}

% custom commands
\newcommand{\shorterSection}[1]{\vspace{-10pt}\section{#1}}

%-------------------------------------------
%%%%%%  CV STARTS HERE  %%%%%%%%%%%%%%%%%%%%%%%%%%%%


\begin{document}

%----------HEADING-----------------
% you can generate your own qr code here: https://www.the-qrcode-generator.com/
% and convert the svg image you exported to pdf here: https://cloudconvert.com/svg-to-pdf
% then import the graph in the title like this:

\begin{table}[]
\begin{tabular*}{\textwidth}{lc@{\extracolsep{\fill}}r}
\begin{tabular}{l}
\textbf{\huge \textcolor{titleblue}{Song Liu}} \\
\\
\end{tabular}  &  & \begin{tabular}{@{}rr@{}} \textcolor{titleblue}
  \quad & \multirow{3}{*}
{\includegraphics[width=0.096\linewidth]{imgs/githubpage.pdf}} \\
\includegraphics[width=0.017\linewidth]{imgs/email(1).pdf} vancirprince@gmail.com                     &                   \\
\includegraphics[width=0.017\linewidth]{imgs/home(1).pdf} \href{https://vancir.github.io/}{vancir.github.io}                            &                   \\
\includegraphics[width=0.017\linewidth]{imgs/phone(1).pdf} (+1) 8143216052                                  &                   
\end{tabular}  \\ 
\end{tabular*}
\end{table}

\vspace*{-10mm}


%-----------EDUCATION-----------------
\shorterSection{\textcolor{titleblue}{Education}}
  \resumeSubHeadingListStart
  \resumeSubheading
      {The Pennsylvania State University, College of Information Sciences and Technology}{PA, USA}     {Doctor of Philosophy}{Aug 2022 - Now}{
      \resumeItemNoBullet{Research Keywords}{Software Security, DBMS Fuzzing, Directed Fuzzing}
    }

    \resumeSubheading
      {Xiamen University, School of Informatics}{Xiamen, China}     {Bachelor of Engineering}{Sep 2015 - Jun 2019}{
    %   \resumeItemNoBullet{Thesis}{Guided Robust Visual Navigation with Deep Learning}
      \resumeItemNoBullet{Relevant Coursework}{Algorithms, Principles of Operating Systems,  Principles of Compilers}
    }
    
  \resumeSubHeadingListEnd



%-----------Addtional Experience & Achievements-----------------
\shorterSection{\textcolor{titleblue}{Publications}}
  \resumeSubHeadingListStart
  \small
  		\item{VIPER: Spotting Syscall-Guard Variables for Data-Only Attacks (\href{https://vancir.github.io/assets/pdf/ye_viper.pdf}{\textcolor{linkblue}{link}}), \textbf{USENIX 2023}}
        \vspace{-5pt}
        
        \item{Can We Trust the Phone Vendors? Comprehensive Security Measurements on the Android Firmware Ecosystem, \textbf{TSE 2023}}
        \vspace{-5pt}
        
  		\item{Detecting Logical Bugs of DBMS with Coverage-based Guidance (\href{https://vancir.github.io/assets/pdf/liang_sqlright.pdf}{\textcolor{linkblue}{link}}), \textbf{USENIX 2022}}
        \vspace{-5pt}
        
        \item{Large-scale Security Measurements on the Android Firmware Ecosystem (\href{https://vancir.github.io/assets/pdf/hou_andscanner.pdf}{\textcolor{linkblue}{link}}), \textbf{ICSE 2022}}
        \vspace{-5pt}
  \resumeSubHeadingListEnd



% %-----------ACADEMIC PROJECTS AND INTERNSHIPS-----------------
% \shorterSection{Academic Projects and internships}
%   \resumeSubHeadingListStart
%   \small
%     \item{
%      \textbf{Languages}{: Python, C++, SQL, Java, Swift}
%      \hfill
%      \textbf{Technologies}{: GCP, AWS, GitHub, GitLab, Docker}
%     }
%     \vspace{-5pt}
%     \item{
%      \textbf{Libraries}{: TensorFlow, PyTorch, Keras, Scikit-Learn, Numpy, Pandas, Spark, Jupyter, OpenCV, PIL, OpenCL, OpenGL, CUDA}
%     }
% \resumeSubHeadingListEnd

%-----------EXPERIENCE-----------------
\shorterSection{\textcolor{titleblue}{Work}}
  \resumeSubHeadingListStart

    \resumeSubheading
      {QI-ANXIN Technology Research Institute}{Beijing, China}
      {Research and Development Engineer, Supervisor: \textbf{Lingyun Ying}}{Aug 2019 - Aug 2022}
      \resumeItemListStart
        \resumeItem{MacOS Sandbox for Malware Analysis}
          { Designed and implemented the \textbf{first} macOS sandbox system in China. For macOS versions below 10.15, I read the source code of \textbf{XNU kernel}, used the Mandatory Access Control Framework(\textbf{MACF}) to monitor \textbf{process behavior} and \textbf{file operations}, wrote Network Kernel Extension(\textbf{NKE}) to capture \textbf{network traffic and behavior} respectively, and use threaded to send the monitoring data from kernel mode to user mode, and wrote a client at user mode to receive the behavior data from kernel mode. For macOS versions 10.15 and above, the previous mechanism was deprecated, so I developed a new system using the EndPoint Security Framework(\textbf{ESF}) and Network Extension(\textbf{NE}) for monitoring and capturing behavioral data. \textbf{Apple Script} was used to write a dynamic traversal tool for macOS apps to simulate user interaction and trigger malicious behavior. }
        \resumeItem{Static and Dynamic Analysis Tool for Android Apps}
          { Developed a static analysis tool based on \textbf{AndroGuard}, with the same analysis capability as \textbf{VirusTotal}. Combined the depth-first search(\textbf{DFS}) algorithm and breadth-first search(\textbf{BFS}) algorithm to traverse the UI of Android App.}
        \resumeItem{Continuous Fuzzing Platform}
          { A \textbf{fuzzing infrastructure} is used to schedule and allocate resources, perform \textbf{continuous} and \textbf{large-scale} fuzzing of target software, and obtain various monitoring data and performance metrics. Various frameworks and middleware are used, such as MongoDB, InfluxDB, Fluentd, Redis, Amazon S3. Integrated multiple \textbf{fuzzing engines}, and supports fuzzing multiple targets on multiple operating system platforms.  }
        \resumeItem{TianWen: Dependency Analysis Platform for Software Supply Chain}
          { Developed a crawler to continuously crawl software binaries. Built and deployed \textbf{graph database cluster}. Optimized database query performance to support querying \textbf{extremely complex} graph information in seconds, and wrote tools to visualize large amounts of graph data.  }
        \resumeItem{CVE Information Extraction Tool}
          { Based on Named Entity Recognition(\textbf{NER}) theory, we use \textbf{BiLSTM-CRF} model to extract the \textbf{vulnerability function}, \textbf{vulnerability version} and \textbf{vulnerability source path} from the unstructured official description information of CVE vulnerabilities, with an accuracy rate of about \textbf{88\%}. }
		\resumeItem{pySnoopSnitch: Android Firmware Patch Existence Detection Tool}
          { Rewrote the \textbf{SnoopSnitch} project code in Python. Supports checking the presence of patches in Android firmware using the \textbf{full core performance} of the server and generating a heat map of patch misses. }
%        \resumeItem{String filter}
%          { A tool for filtering strings dumped from memory which contain a large number of strings that are neither recognizable nor readable by humans, and also contain many duplicate strings that we don't care about that are contained in dependent libraries. I combine \textbf{Shannon entropy} and \textbf{Markov chains} to filter strings, and use \textbf{bloom filters} to filter library strings with excellent time and space efficiency. }
      \resumeItemListEnd

    \resumeSubheading
      {Institute of Information Engineering, Chinese Academy of Sciences}{Beijing, China}
      {Research Intern, Supervisor: \textbf{Feng Li}}{Jul 2018 - Sep 2018}
      \resumeItemListStart
        \resumeItem{IO2BO Vulnerability Automatic Detection}
          { Performed the \textbf{infra-procedural} analysis and \textbf{inter-procedural} analysis on the source code based on \textbf{Klee}. Modified the Klee to read target files as input for \textbf{concolic execution}. When executing, add constraints to the state that may have vulnerabilities and pass the symbolic expression to the \textbf{constraint solver}, and determine whether there is an Integer-Overflow-to-Buffer-Overflow(\textbf{IO2BO}) vulnerability based on the result. }
      \resumeItemListEnd

  \resumeSubHeadingListEnd
  
%-----------PROJECTS/SKILLS-----------------
\shorterSection{\textcolor{titleblue}{Projects}}
  \resumeSubHeadingListStart
    \resumeSubItem{CTF Wiki}
     {
     The CTF Wiki is an open source knowledge base about CTF competitions. I contributed most of the content of the \textbf{reverse engineering} chapter of the Wiki (\href{https://github.com/ctf-wiki/ctf-wiki}{\textcolor{linkblue}{github}}, 7.1k stars)
     }
    \resumeSubItem{Awesome-Binary-Similarity}
      { An awesome list of binary code similarity papers (\href{https://github.com/SystemSecurityStorm/Awesome-Binary-Similarity}{\textcolor{linkblue}{github}}, 359 stars)
	}
    \resumeSubItem{Awesome-Binary-Rewriting}
      { An awesome list of binary rewriting papers (\href{https://github.com/SystemSecurityStorm/Awesome-Binary-Rewriting}{\textcolor{linkblue}{github}}, 172 stars)
	}
	\resumeSubItem{PyPi-Typosquatting-Graph}
     { Analyzing and visualizing \textbf{typosquatting} for Python packages hosted on PyPi.org.  I obtained all python package names and created a \textbf{trie tree}, then calculated the \textbf{edit distance} between the names and used a \textbf{force-directed algorithm} to draw the graph   (\href{https://github.com/Vancir/PyPi-Typosquatting-Graph}{\textcolor{linkblue}{github}})}
    \resumeSubItem{Fugitive Consortium}
      { A blockchain platform based on \textbf{Hyperledger Fabric}. It is written in Java and starts a network topology consisting of docker containers. Chaincode is written to add, delete, query and change information in the ledger  (\href{https://github.com/Vancir/fugitivec}{\textcolor{linkblue}{github}})
	}

  \resumeSubHeadingListEnd
  
  
  
%-------------------------------------------
\end{document}
